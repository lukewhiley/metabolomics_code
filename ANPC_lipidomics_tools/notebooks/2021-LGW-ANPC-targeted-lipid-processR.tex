% Options for packages loaded elsewhere
\PassOptionsToPackage{unicode}{hyperref}
\PassOptionsToPackage{hyphens}{url}
%
\documentclass[
]{article}
\usepackage{amsmath,amssymb}
\usepackage{lmodern}
\usepackage{ifxetex,ifluatex}
\ifnum 0\ifxetex 1\fi\ifluatex 1\fi=0 % if pdftex
  \usepackage[T1]{fontenc}
  \usepackage[utf8]{inputenc}
  \usepackage{textcomp} % provide euro and other symbols
\else % if luatex or xetex
  \usepackage{unicode-math}
  \defaultfontfeatures{Scale=MatchLowercase}
  \defaultfontfeatures[\rmfamily]{Ligatures=TeX,Scale=1}
\fi
% Use upquote if available, for straight quotes in verbatim environments
\IfFileExists{upquote.sty}{\usepackage{upquote}}{}
\IfFileExists{microtype.sty}{% use microtype if available
  \usepackage[]{microtype}
  \UseMicrotypeSet[protrusion]{basicmath} % disable protrusion for tt fonts
}{}
\makeatletter
\@ifundefined{KOMAClassName}{% if non-KOMA class
  \IfFileExists{parskip.sty}{%
    \usepackage{parskip}
  }{% else
    \setlength{\parindent}{0pt}
    \setlength{\parskip}{6pt plus 2pt minus 1pt}}
}{% if KOMA class
  \KOMAoptions{parskip=half}}
\makeatother
\usepackage{xcolor}
\IfFileExists{xurl.sty}{\usepackage{xurl}}{} % add URL line breaks if available
\IfFileExists{bookmark.sty}{\usepackage{bookmark}}{\usepackage{hyperref}}
\hypersetup{
  pdftitle={ANPC SkylineR and Lipid\_exploreR notebook},
  hidelinks,
  pdfcreator={LaTeX via pandoc}}
\urlstyle{same} % disable monospaced font for URLs
\usepackage[margin=1in]{geometry}
\usepackage{longtable,booktabs,array}
\usepackage{calc} % for calculating minipage widths
% Correct order of tables after \paragraph or \subparagraph
\usepackage{etoolbox}
\makeatletter
\patchcmd\longtable{\par}{\if@noskipsec\mbox{}\fi\par}{}{}
\makeatother
% Allow footnotes in longtable head/foot
\IfFileExists{footnotehyper.sty}{\usepackage{footnotehyper}}{\usepackage{footnote}}
\makesavenoteenv{longtable}
\usepackage{graphicx}
\makeatletter
\def\maxwidth{\ifdim\Gin@nat@width>\linewidth\linewidth\else\Gin@nat@width\fi}
\def\maxheight{\ifdim\Gin@nat@height>\textheight\textheight\else\Gin@nat@height\fi}
\makeatother
% Scale images if necessary, so that they will not overflow the page
% margins by default, and it is still possible to overwrite the defaults
% using explicit options in \includegraphics[width, height, ...]{}
\setkeys{Gin}{width=\maxwidth,height=\maxheight,keepaspectratio}
% Set default figure placement to htbp
\makeatletter
\def\fps@figure{htbp}
\makeatother
\setlength{\emergencystretch}{3em} % prevent overfull lines
\providecommand{\tightlist}{%
  \setlength{\itemsep}{0pt}\setlength{\parskip}{0pt}}
\setcounter{secnumdepth}{-\maxdimen} % remove section numbering
\ifluatex
  \usepackage{selnolig}  % disable illegal ligatures
\fi

\title{ANPC SkylineR and Lipid\_exploreR notebook}
\author{}
\date{\vspace{-2.5em}}

\begin{document}
\maketitle

This notebook is designed for use with the ANPC targeted lipid method.
Section 1: SkylineR is designed to optimise lipidomics data processing
in combination with skyline. Section2: Lipid\_exploreR is designed to
explore, visualise and QC check the data.

The sections should be run in sequence. However should section 1 already
be completed, section 2 can be run independently at a later date.

\begin{longtable}[]{@{}
  >{\raggedright\arraybackslash}p{(\columnwidth - 0\tabcolsep) * \real{1.00}}@{}}
\toprule
\endhead
Section 1 - SkylineR \\
This notebook is designed to optimise lipidomics data processing in
combination with skyline. \\
It will perform: - retention time optimisation using LTR QC - peak
boundary fitting to all samples \\
REQUIREMENTS: - A subfolder containig mzML files. Proteowizard should be
used to convert Sciex targeted lipidomics data using proteowizard
default settings - Filename should match the LIMS where possible - mzML
files from LTR samples must have ``LTR'' in their filename - A csv
template containing the target transition details. ONLY the following
column headers should be present: - ``Molecule List'' (lipid family
(e.g.~CE)) - ``Precursor Name'' (lipid name (e.g.~CE(14:0))) -
``Precursor Mz'' (e.g.~614.6) - ``Precursor Charge'' (set as 1) -
``Product Mz'' (e.g.~369.4) - ``Product Charge'' (set as 1) - ``Explicit
Retention Time'' (e.g.~11.66) - ``Explicit Retention Time Window''
(leave at 0.5) - ``Note'' in the column ``Note'' insert the SIL IS to be
used for the target lipid. For the rows containing SIL IS themselves
leave the note column blank. \\
\bottomrule
\end{longtable}

Section 2 - lipid\_exploreR

This notebook is designed to explore the dataset and QC check it. The
script generates a report and a final dataset that can be used for data
modelling.

\hypertarget{lipid-explorer-results}{%
\section{Lipid ExploreR Results}\label{lipid-explorer-results}}

\#notes

multiply by concentration factor report export

\end{document}
